% =============================================================================
% Preamble for Lecture Notes
% Adapted from Gilles Castel's university-setup
% https://castel.dev/post/lecture-notes-1/
% =============================================================================

% --- Core packages ---
\usepackage[utf8]{inputenc}
\usepackage[T1]{fontenc}
\usepackage{textcomp}
\usepackage[english]{babel}
\usepackage{url}
\usepackage{graphicx}
\usepackage{float}
\usepackage{booktabs}
\usepackage{enumitem}

\pdfminorversion=7

% Don't indent paragraphs, leave some space between them
\usepackage{parskip}

% Hide page number when page is empty
\usepackage{emptypage}
\usepackage{subcaption}
\usepackage{multicol}
\usepackage{xcolor}

% --- Math packages ---
\usepackage{amsmath, amsfonts, mathtools, amsthm, amssymb}
\usepackage{mathrsfs}   % Fancy script capitals
\usepackage{cancel}      % Cancel terms in equations
\usepackage{bm}          % Bold math

% --- Common math shortcuts ---
\newcommand\N{\ensuremath{\mathbb{N}}}
\newcommand\R{\ensuremath{\mathbb{R}}}
\newcommand\Z{\ensuremath{\mathbb{Z}}}
\renewcommand\O{\ensuremath{\emptyset}}
\newcommand\Q{\ensuremath{\mathbb{Q}}}
\newcommand\C{\ensuremath{\mathbb{C}}}
\newcommand\F{\ensuremath{\mathbb{F}}}

% Put x \to \infty below \lim
\let\svlim\lim\def\lim{\svlim\limits}

% Shorter implies/iff arrows
\let\implies\Rightarrow
\let\impliedby\Leftarrow
\let\iff\Leftrightarrow
\let\epsilon\varepsilon

% Contradiction symbol
\usepackage{stmaryrd}
\newcommand\contra{\scalebox{1.5}{$\lightning$}}

% Correction markup: \correct{wrong}{right}
\definecolor{correct}{HTML}{009900}
\newcommand\correct[2]{\ensuremath{\:}{\color{red}{#1}}\ensuremath{\to }{\color{correct}{#2}}\ensuremath{\:}}
\newcommand\green[1]{{\color{correct}{#1}}}

% Horizontal rule
\newcommand\hr{\noindent\rule[0.5ex]{\linewidth}{0.5pt}}

% SI units
\usepackage{siunitx}
\sisetup{locale = US}

% Systems of equations
\usepackage{systeme}

% --- Theorem-like environments (boxed) ---
\makeatother
\usepackage{mdframed}
\mdfsetup{skipabove=1em,skipbelow=0em}

\theoremstyle{definition}

% Boxed (numbered) environments
\newmdtheoremenv[nobreak=true]{definition}{Definition}[section]
\newmdtheoremenv[nobreak=true]{theorem}{Theorem}[section]
\newmdtheoremenv[nobreak=true]{lemma}{Lemma}[section]
\newmdtheoremenv[nobreak=true]{corollary}{Corollary}[section]
\newmdtheoremenv[nobreak=true]{prop}{Proposition}[section]
\newmdtheoremenv[nobreak=true]{axiom}{Axiom}[section]
\newmdtheoremenv{conjecture}{Conjecture}[section]

% Non-boxed (unnumbered) environments
\newtheorem*{eg}{Example}
\newtheorem*{notation}{Notation}
\newtheorem*{previouslyseen}{As previously seen}
\newtheorem*{remark}{Remark}
\newtheorem*{note}{Note}
\newtheorem*{problem}{Problem}
\newtheorem*{observe}{Observe}
\newtheorem*{property}{Property}
\newtheorem*{intuition}{Intuition}
\newtheorem*{recall}{Recall}
\newtheorem*{exercise}{Exercise}

% End example environments with a diamond
\usepackage{etoolbox}
\AtEndEnvironment{eg}{\null\hfill$\diamond$}%

% Fix spacing around theorem environments
\makeatletter
\def\thm@space@setup{%
  \thm@preskip=\parskip \thm@postskip=0pt
}

% --- Lecture header command ---
% Usage: \lecture{1}{Mon 03 Feb 2025 10:00}{Introduction to Group Theory}
\usepackage{xifthen}
\def\testdateparts#1{\dateparts#1\relax}
\def\dateparts#1 #2 #3 #4 #5\relax{
    \marginpar{\small\textsf{\mbox{#1 #2 #3 #5}}}
}

\def\@lecture{}%
\newcommand{\lecture}[3]{
    \ifthenelse{\isempty{#3}}{%
        \def\@lecture{Lecture #1}%
    }{%
        \def\@lecture{Lecture #1: #3}%
    }%
    \subsection*{\@lecture}
    \marginpar{\small\textsf{\mbox{#2}}}
}

% --- Fancy headers ---
\usepackage{fancyhdr}
\pagestyle{fancy}

\fancyhead[RO,LE]{\@lecture}
\fancyhead[RE,LO]{}
\fancyfoot[RO,LE]{\thepage}
\fancyfoot[RE,LO]{}
\fancyfoot[C]{\leftmark}

\makeatother

% --- Note / correction boxes ---
\usepackage{todonotes}
\usepackage{tcolorbox}
\tcbuselibrary{breakable}

% Correction box (green border)
\newenvironment{correction}{\begin{tcolorbox}[
    arc=0mm,
    colback=white,
    colframe=green!60!black,
    title=Correction,
    fonttitle=\sffamily,
    breakable
]}{\end{tcolorbox}}

% Generic note box
\newenvironment{notebox}[1]{\begin{tcolorbox}[
    arc=0mm,
    colback=white,
    colframe=white!60!black,
    title=#1,
    fonttitle=\sffamily,
    breakable
]}{\end{tcolorbox}}

% Important box (red border)
\newenvironment{important}[1]{\begin{tcolorbox}[
    arc=0mm,
    colback=red!5,
    colframe=red!60!black,
    title=#1,
    fonttitle=\sffamily,
    breakable
]}{\end{tcolorbox}}

% --- Figure support (Inkscape integration) ---
\usepackage{import}
\usepackage{pdfpages}
\usepackage{transparent}
\newcommand{\incfig}[1]{%
    \def\svgwidth{\columnwidth}
    \import{./figures/}{#1.pdf_tex}
}

% Suppress PDF page group warnings
\pdfsuppresswarningpagegroup=1

% --- TikZ support ---
\usepackage{tikz}
\usepackage{tikz-cd}       % Commutative diagrams
\usepackage{pgfplots}       % Plots and graphs
\pgfplotsset{compat=1.18}
\usetikzlibrary{
    arrows.meta,
    calc,
    decorations.markings,
    decorations.pathreplacing,
    intersections,
    patterns,
    positioning,
    shapes.geometric,
    angles,
    quotes,
    babel
}

% --- Hyperlinks (load last) ---
\usepackage{hyperref}
\hypersetup{
    colorlinks=true,
    linkcolor=blue!60!black,
    urlcolor=blue!60!black,
    citecolor=green!60!black,
}
\usepackage[nameinlink]{cleveref}
